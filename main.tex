% arara: lualatex: {options: []}
% arara: biber: { options: [ '--wraplines' ] }
% arara: lualatex: {options: []}
\documentclass[10pt]{article}
\usepackage{notestex,csquotes,tabularx,lipsum}
\SetColorProfile{9C8C52}{BEA035}{3D4CDB}{525786}{5C5748}
\usepackage[ngerman]{babel}
\usepackage[
backend=biber,
style=alphabetic,
sorting=ynt
]{biblatex}
\renewcommand{\arraystretch}{1.5}
\usetikzlibrary{mindmap,trees}
\tikzset{
concept/.append style={fill={none}, text=black, font=\sffamily\footnotesize},
root concept/.append style={concept color=ntex@color@primary},
level 1 concept/.append style={every child/.style={concept color=ntex@color@secondary}},
level 2 concept/.append style={every child/.style={concept color=ntex@color@tertiary}},
level 3 concept/.append style={every child/.style={concept color=ntex@color@quaternary}},
level 4 concept/.append style={every child/.style={concept color=ntexq@color@quinary}},
}

\begin{document}
\title{{Englisch}\\{\normalsize{\itshape Leistungskurs}}}
	\author{Never Gude}
	\affiliation{
	Carl-Friedrich-Gauß Gymnasium Hockenheim\\
	\href{https://github.com/minced1}{GitHub}\\
	}
	\emailAdd{j.gude/posteo.de}
	\license{This work is licensed under \href{https://creativecommons.org/licenses/by-nc-sa/4.0/}{CC BY-NC-SA 4.0}}{\ccbyncsa}
\maketitle
\newpage

	% \usetikzlibrary {arrows.meta,graphs,graphdrawing}
	% \usegdlibrary {circular}
	% \begin{tikzpicture}
\newcommand\myref[1]{\footnotesize \nameref{#1}}
	% \centering
	% \graph [
	% simple necklace layout,
	% nodes={
	% 	draw, align=center, inner xsep=0.5em, text width=2cm,
	% 	typeset={\myref{\tikzgraphnodefullname}}
	% },
	% ] { uk --  {uk/empire, goal, uk/immig -- {goal -- uk, uk/empire}} };
 %  \end{tikzpicture}


\pagestyle{empty}

\newpage
\pagestyle{fancynotes}

\part{Abiturprüfung}
\section{Rechnerstrukturen}
\subsection{Logikgatter}
\usetikzlibrary {circuits.logic.IEC}

\begin{table}[htbp]
	\centering
	\begin{tabularx}{\textwidth}{X X X}
		\hline
		{\sffamily\bfseries IEC Symbol} & {\sffamily\bfseries Mathematisches Symbol} & {\sffamily\bfseries Bedeutung} \\ \hline
		\tikz[circuit logic IEC]{\node [not gate] (a1) {}; } & $\neg$ & NOT/Negation \\
		\tikz[circuit logic IEC]{\node [and gate] (a1) {}; } & $\land$ & AND/Konjunktion \\
		\tikz[circuit logic IEC]{\node [or gate] (a1) {}; } & $\lor$ & OR/Disjunktion \\
		\tikz[circuit logic IEC]{\node [xor gate] (a1) {}; } & $\oplus$ & XOR/Kontravalenz \\
		\hline
	\end{tabularx}
	\caption{Logikgatter mit deren mathematischen Symbolen}
	\label{tab:logik}
\end{table}

\begin{figure}[htbp]
\caption{Halbaddierer}
\begin{tikzpicture}[circuit logic IEC]
  \matrix[column sep=7mm]
  {
		\node (i0) {0}; & & \node [and gate] (a1) {};  & \node (c) {Übertrag c}; \\
    & \node (i1) {0}; &                             & \\
    &                & \node [xor gate] (a2) {}; & \node (s) {Summe s}; \\
    };
		\draw (i0.east) -- ++(right:3mm) |- (a1.input 1);
		\draw (i0.east) -- ++(right:3mm) |- (a2.input 1);
		\draw (i1.east) -- ++(right:3mm) |- (a1.input 2);
		\draw (i1.east) -- ++(right:3mm) |- (a2.input 2);
		\draw (a1.output) -- ++(right:3mm) |- (c);
		\draw (a2.output) -- ++(right:3mm) |- (s);
	\end{tikzpicture}
	\label{fig:halbadd}
\end{figure}

\begin{figure}[htbp]
\caption{Volladdierer}
\begin{tikzpicture}[circuit logic IEC]
  \matrix[column sep=7mm]
  {
		\node (i0) {Eingang a}; & & \node [and gate, and gate IEC symbol = \char`\H,] (a1) {}; & & \node (s) {Summe s}; \\
    \node (i1) {Eingang b}; & \node [and gate, and gate IEC symbol = \char`\H,] (a2) {}; & & & \\
    \node (i2) {Übertrag c}; &                & & \node [or gate] (a3) {}; & \node (c1) {Übertrag c};\\
    };
		\draw (i0.east) -- ++(right:3mm) |- (a1.input 1);
		%\draw (i0.east) -- ++(right:3mm) |- (a2.input 1);
		\draw (i1.east) -- ++(right:3mm) |- (a2.input 1);
		\draw (i2.east) -- ++(right:3mm) |- (a2.input 2);
		\draw (a1.output) -- ++(right:3mm) |- (s);
		\draw (a1.output) -- ++(right:3mm) |- (a3.input 1);
		\draw (a2.output) -- ++(right:3mm) |- (a1.input 2);
		\draw (a2.output) -- ++(right:3mm) |- (a3.input 2);
		\draw (a3.output) -- ++(right:3mm) |- (c1);
	\end{tikzpicture}
	\label{fig:volladd}
\end{figure}

\begin{figure}[htbp]
\caption{RS-Flip-Flop}
\begin{tikzpicture}[circuit logic IEC]
  \matrix[column sep=7mm]
  {
		\node (i0) {0}; & & & \node [or gate] (a1) {}; & \node (q) {Ausgang q}; \\
    \node (i1) {0}; & \node [not gate] (a2) {}; & \node [and gate] (a3) {}; & &\\
    };
		\draw (i0.east) -- ++(right:3mm) |- (a1.input 1);
		\draw (i1.east) -- ++(right:3mm) |- (a2.input);
		\draw (a1.output) -- ++(right:3mm) |- (q);
		\draw (a2.output) -- ++(right:3mm) |- (a3.input 2);
		\draw (a3.output) -- ++(right:3mm) |- (a1.input 2);
		\draw (a1.output) -- ++(right:3mm) |- (a3.input 1);
	\end{tikzpicture}
	\label{fig:halbadd}
\end{figure}

\subsection{Boolsche Algebra}
\paragraph{KV-Diagramme}

\section{Netzwerke}
\subsection{Schichtenmodell}

\section{Krytologie}
\subsection{Kerckhoffsches Prinzip}
Im Jahr 1883 formulierte Auguste Kerckhoff ein Grundprinzip der Kryptologie:
\begin{quote}
In einem guten Kryptosystem muss nur der Schlüssel geheim bleiben
\end{quote}
\subsection{Ziele der Verschlüsselung}
\paragraph{Vertraulichkeit}
Um ein Mitlesen der Nachricht durch Dritte zu verhindern, muss die Vertraulichkeit gewährt sein.
\paragraph{Integrität}
Um ein Ändern der Nachricht durch Dritte zu verhindern, muss die Integrität gewährt sein.
\paragraph{Authentizität}
Um ein Ändern des*der Absender*in zu verhindern, muss die Authentizität gewährt sein.
\subsection{(A)symmetrische Verschlüsselung}
\paragraph{Symmetrische Verschlüsselung} Die Nachricht wird mit dem gleichen Schlüssel ver- und entschlüsselt. Der Rechenaufwand ist geringer als beim asymmetrischen Pendent.\\
Bsp.: AES
\paragraph{Asymmetrische Verschlüsselung} Die Nachricht wird mit einem Schlüssel verschlüsselt und mit einem anderen entschlüsselt. Es wird dabei ein Schlüsselpaar aus öffentlichem und privatem Schlüssel generiert.

Mit dem fremden öffentlichen Schlüssel wird verschlüsselt, wenn Vertraulichkeit gefordert ist. Entschlüsselt wird dann mit dem eigenen privaten Schlüssel.

Mit dem eigenen privaten Schlüssel wird verschlüsselt, wenn Authentizität gefordert ist. Entschlüsselt wird dann mit dem fremden öffentlichen Schlüssel. Dieses Verfahren wird in der
Realität häufig praktiziert.
\\
Bsp.: RSA
\paragraph{Hybride Verschlüsselung} Der symmetrische Schlüssel wird per asymmetrischer Verschlüsselung an die andere Person gesendet. Danach kann der symmetrische Schlüssel für die weitere Kommunikation verwendet werden.
\paragraph{Transpositionsverfahren} Die Buchstaben des Klartextes werden durch die Verschlüsselung anders angeordnet.\\
Bsp.: Skytale, (Fleißner, Gartenzaun)
\paragraph{Substitutionsverfahren} Die Buchstaben des Klartextes werden bei der Verschlüsselung durch andere Buchstaben (oder Zeichen) ersetzt.\\
Bsp.: Caesar, Substitutionschiffre, Vigenère, One-Time-Pad
\paragraph{Monoalphabetische Substitution} Ein Substitutionsverfahren, bei dem nur ein einziges Schlüsselalphabet verwendet wird.\\
Bsp.: Caesar, RSA (wenn man es fälschlicherweise auf einzelne Buchstaben anwendet)
\paragraph{Polyalphabetische Substitution} Ein Substitutionsverfahren, bei dem mehrere
Schlüsselalphabete verwendet werden.\\
Bsp.: Vigenère, One-Time-Pad
\paragraph{One-Time-Pad} Absolut sichere Version der Vigenère-Verschlüsselung, bei dem der Schlüssel mindestens so lang sein muss, wie die Nachricht. Außerdem muss der Schlüssel sicher übermittelt werden und nur einmal verwendet werden. Da bei der sicheren Übertragung des Schlüssels auch gleich die Nachricht übermittelt werden könnte, ist das OTP unpraktikabel.
\printntexbibliography
\end{document}
